\documentclass[]{article}
\usepackage{lmodern}
\usepackage{amssymb,amsmath}
\usepackage{ifxetex,ifluatex}
\usepackage{fixltx2e} % provides \textsubscript
\ifnum 0\ifxetex 1\fi\ifluatex 1\fi=0 % if pdftex
  \usepackage[T1]{fontenc}
  \usepackage[utf8]{inputenc}
\else % if luatex or xelatex
  \ifxetex
    \usepackage{mathspec}
  \else
    \usepackage{fontspec}
  \fi
  \defaultfontfeatures{Ligatures=TeX,Scale=MatchLowercase}
\fi
% use upquote if available, for straight quotes in verbatim environments
\IfFileExists{upquote.sty}{\usepackage{upquote}}{}
% use microtype if available
\IfFileExists{microtype.sty}{%
\usepackage[]{microtype}
\UseMicrotypeSet[protrusion]{basicmath} % disable protrusion for tt fonts
}{}
\PassOptionsToPackage{hyphens}{url} % url is loaded by hyperref
\usepackage[unicode=true]{hyperref}
\hypersetup{
            pdftitle={Dataframes na Linguagem R},
            pdfauthor={Ivanildo Batista},
            pdfborder={0 0 0},
            breaklinks=true}
\urlstyle{same}  % don't use monospace font for urls
\usepackage[margin=1in]{geometry}
\usepackage{color}
\usepackage{fancyvrb}
\newcommand{\VerbBar}{|}
\newcommand{\VERB}{\Verb[commandchars=\\\{\}]}
\DefineVerbatimEnvironment{Highlighting}{Verbatim}{commandchars=\\\{\}}
% Add ',fontsize=\small' for more characters per line
\usepackage{framed}
\definecolor{shadecolor}{RGB}{248,248,248}
\newenvironment{Shaded}{\begin{snugshade}}{\end{snugshade}}
\newcommand{\KeywordTok}[1]{\textcolor[rgb]{0.13,0.29,0.53}{\textbf{#1}}}
\newcommand{\DataTypeTok}[1]{\textcolor[rgb]{0.13,0.29,0.53}{#1}}
\newcommand{\DecValTok}[1]{\textcolor[rgb]{0.00,0.00,0.81}{#1}}
\newcommand{\BaseNTok}[1]{\textcolor[rgb]{0.00,0.00,0.81}{#1}}
\newcommand{\FloatTok}[1]{\textcolor[rgb]{0.00,0.00,0.81}{#1}}
\newcommand{\ConstantTok}[1]{\textcolor[rgb]{0.00,0.00,0.00}{#1}}
\newcommand{\CharTok}[1]{\textcolor[rgb]{0.31,0.60,0.02}{#1}}
\newcommand{\SpecialCharTok}[1]{\textcolor[rgb]{0.00,0.00,0.00}{#1}}
\newcommand{\StringTok}[1]{\textcolor[rgb]{0.31,0.60,0.02}{#1}}
\newcommand{\VerbatimStringTok}[1]{\textcolor[rgb]{0.31,0.60,0.02}{#1}}
\newcommand{\SpecialStringTok}[1]{\textcolor[rgb]{0.31,0.60,0.02}{#1}}
\newcommand{\ImportTok}[1]{#1}
\newcommand{\CommentTok}[1]{\textcolor[rgb]{0.56,0.35,0.01}{\textit{#1}}}
\newcommand{\DocumentationTok}[1]{\textcolor[rgb]{0.56,0.35,0.01}{\textbf{\textit{#1}}}}
\newcommand{\AnnotationTok}[1]{\textcolor[rgb]{0.56,0.35,0.01}{\textbf{\textit{#1}}}}
\newcommand{\CommentVarTok}[1]{\textcolor[rgb]{0.56,0.35,0.01}{\textbf{\textit{#1}}}}
\newcommand{\OtherTok}[1]{\textcolor[rgb]{0.56,0.35,0.01}{#1}}
\newcommand{\FunctionTok}[1]{\textcolor[rgb]{0.00,0.00,0.00}{#1}}
\newcommand{\VariableTok}[1]{\textcolor[rgb]{0.00,0.00,0.00}{#1}}
\newcommand{\ControlFlowTok}[1]{\textcolor[rgb]{0.13,0.29,0.53}{\textbf{#1}}}
\newcommand{\OperatorTok}[1]{\textcolor[rgb]{0.81,0.36,0.00}{\textbf{#1}}}
\newcommand{\BuiltInTok}[1]{#1}
\newcommand{\ExtensionTok}[1]{#1}
\newcommand{\PreprocessorTok}[1]{\textcolor[rgb]{0.56,0.35,0.01}{\textit{#1}}}
\newcommand{\AttributeTok}[1]{\textcolor[rgb]{0.77,0.63,0.00}{#1}}
\newcommand{\RegionMarkerTok}[1]{#1}
\newcommand{\InformationTok}[1]{\textcolor[rgb]{0.56,0.35,0.01}{\textbf{\textit{#1}}}}
\newcommand{\WarningTok}[1]{\textcolor[rgb]{0.56,0.35,0.01}{\textbf{\textit{#1}}}}
\newcommand{\AlertTok}[1]{\textcolor[rgb]{0.94,0.16,0.16}{#1}}
\newcommand{\ErrorTok}[1]{\textcolor[rgb]{0.64,0.00,0.00}{\textbf{#1}}}
\newcommand{\NormalTok}[1]{#1}
\usepackage{graphicx,grffile}
\makeatletter
\def\maxwidth{\ifdim\Gin@nat@width>\linewidth\linewidth\else\Gin@nat@width\fi}
\def\maxheight{\ifdim\Gin@nat@height>\textheight\textheight\else\Gin@nat@height\fi}
\makeatother
% Scale images if necessary, so that they will not overflow the page
% margins by default, and it is still possible to overwrite the defaults
% using explicit options in \includegraphics[width, height, ...]{}
\setkeys{Gin}{width=\maxwidth,height=\maxheight,keepaspectratio}
\IfFileExists{parskip.sty}{%
\usepackage{parskip}
}{% else
\setlength{\parindent}{0pt}
\setlength{\parskip}{6pt plus 2pt minus 1pt}
}
\setlength{\emergencystretch}{3em}  % prevent overfull lines
\providecommand{\tightlist}{%
  \setlength{\itemsep}{0pt}\setlength{\parskip}{0pt}}
\setcounter{secnumdepth}{0}
% Redefines (sub)paragraphs to behave more like sections
\ifx\paragraph\undefined\else
\let\oldparagraph\paragraph
\renewcommand{\paragraph}[1]{\oldparagraph{#1}\mbox{}}
\fi
\ifx\subparagraph\undefined\else
\let\oldsubparagraph\subparagraph
\renewcommand{\subparagraph}[1]{\oldsubparagraph{#1}\mbox{}}
\fi

% set default figure placement to htbp
\makeatletter
\def\fps@figure{htbp}
\makeatother


\title{Dataframes na Linguagem R}
\author{Ivanildo Batista}
\date{27 de janeiro de 2021}

\begin{document}
\maketitle

\subsection{Dataframes}\label{dataframes}

Trabalhando com \textbf{dataframes} na linguagem R.

Abaixo, iniciarei criando um dataframe vazio e depois verificando sua
classe:

\begin{Shaded}
\begin{Highlighting}[]
\NormalTok{df=}\KeywordTok{data.frame}\NormalTok{() }\CommentTok{#um dataframe vazio}
\KeywordTok{class}\NormalTok{(df)}\CommentTok{#verificando a classe}
\end{Highlighting}
\end{Shaded}

\begin{verbatim}
## [1] "data.frame"
\end{verbatim}

\begin{Shaded}
\begin{Highlighting}[]
\NormalTok{df }\CommentTok{#sem linhas e colunas}
\end{Highlighting}
\end{Shaded}

\begin{verbatim}
## data frame with 0 columns and 0 rows
\end{verbatim}

\textbf{Criando vetores vazios:}

\begin{Shaded}
\begin{Highlighting}[]
\NormalTok{nomes =}\StringTok{ }\KeywordTok{character}\NormalTok{() }\CommentTok{#vetor de caracteres}
\NormalTok{idades =}\StringTok{ }\KeywordTok{numeric}\NormalTok{() }\CommentTok{#vetor de numeros}
\NormalTok{data =}\StringTok{ }\KeywordTok{as.Date}\NormalTok{(}\KeywordTok{character}\NormalTok{()) }\CommentTok{#vetor de datas}
\NormalTok{codigos =}\StringTok{ }\KeywordTok{integer}\NormalTok{() }\CommentTok{#vetor de inteiros}
\end{Highlighting}
\end{Shaded}

\textbf{Transformando esses vetores em um dataframe:}

\begin{Shaded}
\begin{Highlighting}[]
\NormalTok{df =}\StringTok{ }\KeywordTok{data.frame}\NormalTok{(}\KeywordTok{c}\NormalTok{(nomes,idades,data,codigos))}
\NormalTok{df}
\end{Highlighting}
\end{Shaded}

\begin{verbatim}
## [1] c.nomes..idades..data..codigos.
## <0 rows> (or 0-length row.names)
\end{verbatim}

\textbf{Criando vetores:}

\begin{Shaded}
\begin{Highlighting}[]
\NormalTok{pais =}\StringTok{ }\KeywordTok{c}\NormalTok{(}\StringTok{'EUA'}\NormalTok{,}\StringTok{'Dinamarca'}\NormalTok{,}\StringTok{'Brasil'}\NormalTok{,}\StringTok{'Espanha'}\NormalTok{,}\StringTok{'Holanda'}\NormalTok{)}
\NormalTok{nome =}\StringTok{ }\KeywordTok{c}\NormalTok{(}\StringTok{'Alberto'}\NormalTok{,}\StringTok{'Claudio'}\NormalTok{,}\StringTok{'Joao'}\NormalTok{,}\StringTok{'Jose'}\NormalTok{,}\StringTok{'Gilberto'}\NormalTok{)}
\NormalTok{altura =}\StringTok{ }\KeywordTok{c}\NormalTok{(}\FloatTok{1.93}\NormalTok{,}\FloatTok{1.59}\NormalTok{,}\FloatTok{1.63}\NormalTok{,}\FloatTok{1.59}\NormalTok{,}\FloatTok{1.58}\NormalTok{)}
\NormalTok{codigo =}\StringTok{ }\KeywordTok{c}\NormalTok{(}\DecValTok{5069}\NormalTok{,}\DecValTok{3258}\NormalTok{,}\DecValTok{6358}\NormalTok{,}\DecValTok{1258}\NormalTok{,}\DecValTok{4555}\NormalTok{)}
\end{Highlighting}
\end{Shaded}

\textbf{Criando o dataframe:}

\begin{Shaded}
\begin{Highlighting}[]
\NormalTok{pesq =}\StringTok{ }\KeywordTok{data.frame}\NormalTok{(pais,nome,altura,codigo)}
\NormalTok{pesq}
\end{Highlighting}
\end{Shaded}

\begin{verbatim}
##        pais     nome altura codigo
## 1       EUA  Alberto   1.93   5069
## 2 Dinamarca  Claudio   1.59   3258
## 3    Brasil     Joao   1.63   6358
## 4   Espanha     Jose   1.59   1258
## 5   Holanda Gilberto   1.58   4555
\end{verbatim}

\textbf{Inserindo uma nova coluna:}

\begin{Shaded}
\begin{Highlighting}[]
\NormalTok{olhos =}\StringTok{ }\KeywordTok{c}\NormalTok{(}\StringTok{'verde'}\NormalTok{,}\StringTok{'castanho'}\NormalTok{,}\StringTok{'azul'}\NormalTok{,}\StringTok{'azul'}\NormalTok{,}\StringTok{'castanho'}\NormalTok{)}
\NormalTok{pesq =}\StringTok{ }\KeywordTok{cbind}\NormalTok{(pesq,olhos)}
\NormalTok{pesq}
\end{Highlighting}
\end{Shaded}

\begin{verbatim}
##        pais     nome altura codigo    olhos
## 1       EUA  Alberto   1.93   5069    verde
## 2 Dinamarca  Claudio   1.59   3258 castanho
## 3    Brasil     Joao   1.63   6358     azul
## 4   Espanha     Jose   1.59   1258     azul
## 5   Holanda Gilberto   1.58   4555 castanho
\end{verbatim}

\textbf{Informacoes sobre o dataframe:}

\begin{Shaded}
\begin{Highlighting}[]
\KeywordTok{str}\NormalTok{(pesq) }\CommentTok{#informacoes gerais do dataframe}
\end{Highlighting}
\end{Shaded}

\begin{verbatim}
## 'data.frame':    5 obs. of  5 variables:
##  $ pais  : chr  "EUA" "Dinamarca" "Brasil" "Espanha" ...
##  $ nome  : chr  "Alberto" "Claudio" "Joao" "Jose" ...
##  $ altura: num  1.93 1.59 1.63 1.59 1.58
##  $ codigo: num  5069 3258 6358 1258 4555
##  $ olhos : chr  "verde" "castanho" "azul" "azul" ...
\end{verbatim}

\begin{Shaded}
\begin{Highlighting}[]
\KeywordTok{dim}\NormalTok{(pesq) }\CommentTok{#dimnesao do dataframe}
\end{Highlighting}
\end{Shaded}

\begin{verbatim}
## [1] 5 5
\end{verbatim}

\begin{Shaded}
\begin{Highlighting}[]
\KeywordTok{length}\NormalTok{(pesq) }\CommentTok{#comprimento do dataframe}
\end{Highlighting}
\end{Shaded}

\begin{verbatim}
## [1] 5
\end{verbatim}

\textbf{Obtendo um vetor do dataframe}

\begin{Shaded}
\begin{Highlighting}[]
\NormalTok{pesq}\OperatorTok{$}\NormalTok{pais}
\end{Highlighting}
\end{Shaded}

\begin{verbatim}
## [1] "EUA"       "Dinamarca" "Brasil"    "Espanha"   "Holanda"
\end{verbatim}

\begin{Shaded}
\begin{Highlighting}[]
\NormalTok{pesq}\OperatorTok{$}\NormalTok{altura}
\end{Highlighting}
\end{Shaded}

\begin{verbatim}
## [1] 1.93 1.59 1.63 1.59 1.58
\end{verbatim}

\textbf{Extraindo um valor unico do dataframe}

\begin{Shaded}
\begin{Highlighting}[]
\NormalTok{pesq[}\DecValTok{1}\NormalTok{,}\DecValTok{1}\NormalTok{]}
\end{Highlighting}
\end{Shaded}

\begin{verbatim}
## [1] "EUA"
\end{verbatim}

\begin{Shaded}
\begin{Highlighting}[]
\NormalTok{pesq[}\DecValTok{3}\NormalTok{,}\DecValTok{2}\NormalTok{]}
\end{Highlighting}
\end{Shaded}

\begin{verbatim}
## [1] "Joao"
\end{verbatim}

\textbf{Numero de linhas e de colunas do dataframe}

\begin{Shaded}
\begin{Highlighting}[]
\KeywordTok{nrow}\NormalTok{(pesq)}
\end{Highlighting}
\end{Shaded}

\begin{verbatim}
## [1] 5
\end{verbatim}

\begin{Shaded}
\begin{Highlighting}[]
\KeywordTok{ncol}\NormalTok{(pesq)}
\end{Highlighting}
\end{Shaded}

\begin{verbatim}
## [1] 5
\end{verbatim}

\textbf{Primeiros elementos do dataframe:}

\begin{Shaded}
\begin{Highlighting}[]
\KeywordTok{head}\NormalTok{(pesq)}
\end{Highlighting}
\end{Shaded}

\begin{verbatim}
##        pais     nome altura codigo    olhos
## 1       EUA  Alberto   1.93   5069    verde
## 2 Dinamarca  Claudio   1.59   3258 castanho
## 3    Brasil     Joao   1.63   6358     azul
## 4   Espanha     Jose   1.59   1258     azul
## 5   Holanda Gilberto   1.58   4555 castanho
\end{verbatim}

\begin{Shaded}
\begin{Highlighting}[]
\KeywordTok{head}\NormalTok{(mtcars)}
\end{Highlighting}
\end{Shaded}

\begin{verbatim}
##                    mpg cyl disp  hp drat    wt  qsec vs am gear carb
## Mazda RX4         21.0   6  160 110 3.90 2.620 16.46  0  1    4    4
## Mazda RX4 Wag     21.0   6  160 110 3.90 2.875 17.02  0  1    4    4
## Datsun 710        22.8   4  108  93 3.85 2.320 18.61  1  1    4    1
## Hornet 4 Drive    21.4   6  258 110 3.08 3.215 19.44  1  0    3    1
## Hornet Sportabout 18.7   8  360 175 3.15 3.440 17.02  0  0    3    2
## Valiant           18.1   6  225 105 2.76 3.460 20.22  1  0    3    1
\end{verbatim}

\textbf{Ultimos elementos do dataframe:}

\begin{Shaded}
\begin{Highlighting}[]
\KeywordTok{tail}\NormalTok{(pesq)}
\end{Highlighting}
\end{Shaded}

\begin{verbatim}
##        pais     nome altura codigo    olhos
## 1       EUA  Alberto   1.93   5069    verde
## 2 Dinamarca  Claudio   1.59   3258 castanho
## 3    Brasil     Joao   1.63   6358     azul
## 4   Espanha     Jose   1.59   1258     azul
## 5   Holanda Gilberto   1.58   4555 castanho
\end{verbatim}

\begin{Shaded}
\begin{Highlighting}[]
\KeywordTok{tail}\NormalTok{(mtcars)}
\end{Highlighting}
\end{Shaded}

\begin{verbatim}
##                 mpg cyl  disp  hp drat    wt qsec vs am gear carb
## Porsche 914-2  26.0   4 120.3  91 4.43 2.140 16.7  0  1    5    2
## Lotus Europa   30.4   4  95.1 113 3.77 1.513 16.9  1  1    5    2
## Ford Pantera L 15.8   8 351.0 264 4.22 3.170 14.5  0  1    5    4
## Ferrari Dino   19.7   6 145.0 175 3.62 2.770 15.5  0  1    5    6
## Maserati Bora  15.0   8 301.0 335 3.54 3.570 14.6  0  1    5    8
## Volvo 142E     21.4   4 121.0 109 4.11 2.780 18.6  1  1    4    2
\end{verbatim}

\textbf{Filtrando o dataframe}

Irei filtrar para selecionar todas as linhas onde o valor de altura seja
menor que 1.60 metro.

\begin{Shaded}
\begin{Highlighting}[]
\NormalTok{pesq[altura}\OperatorTok{<}\FloatTok{1.60}\NormalTok{]}
\end{Highlighting}
\end{Shaded}

\begin{verbatim}
##       nome codigo    olhos
## 1  Alberto   5069    verde
## 2  Claudio   3258 castanho
## 3     Joao   6358     azul
## 4     Jose   1258     azul
## 5 Gilberto   4555 castanho
\end{verbatim}

\begin{Shaded}
\begin{Highlighting}[]
\NormalTok{pesq[altura}\OperatorTok{<}\FloatTok{1.60}\NormalTok{, }\KeywordTok{c}\NormalTok{(}\StringTok{'codigo'}\NormalTok{,}\StringTok{'olhos'}\NormalTok{)] }\CommentTok{#selecionada as colunas codigo e olhos,}
\end{Highlighting}
\end{Shaded}

\begin{verbatim}
##   codigo    olhos
## 2   3258 castanho
## 4   1258     azul
## 5   4555 castanho
\end{verbatim}

\begin{Shaded}
\begin{Highlighting}[]
\CommentTok{#para os valores de altura menores que 1.60}
\end{Highlighting}
\end{Shaded}

\textbf{Nomeando dataframes}

\begin{Shaded}
\begin{Highlighting}[]
\KeywordTok{names}\NormalTok{(pesq) =}\StringTok{ }\KeywordTok{c}\NormalTok{(}\StringTok{'Pais'}\NormalTok{,}\StringTok{'Nome'}\NormalTok{,}\StringTok{'Altura'}\NormalTok{,}\StringTok{'Codigos'}\NormalTok{,}\StringTok{'Olhos'}\NormalTok{)}
\NormalTok{pesq}
\end{Highlighting}
\end{Shaded}

\begin{verbatim}
##        Pais     Nome Altura Codigos    Olhos
## 1       EUA  Alberto   1.93    5069    verde
## 2 Dinamarca  Claudio   1.59    3258 castanho
## 3    Brasil     Joao   1.63    6358     azul
## 4   Espanha     Jose   1.59    1258     azul
## 5   Holanda Gilberto   1.58    4555 castanho
\end{verbatim}

\textbf{Renomeando apenas colunas}

\begin{Shaded}
\begin{Highlighting}[]
\KeywordTok{colnames}\NormalTok{(pesq) =}\StringTok{ }\KeywordTok{c}\NormalTok{(}\StringTok{'Var 1'}\NormalTok{,}\StringTok{'Var 2'}\NormalTok{,}\StringTok{'Var 3'}\NormalTok{,}\StringTok{'Var 4'}\NormalTok{,}\StringTok{'Var 5'}\NormalTok{) }
\NormalTok{pesq}
\end{Highlighting}
\end{Shaded}

\begin{verbatim}
##       Var 1    Var 2 Var 3 Var 4    Var 5
## 1       EUA  Alberto  1.93  5069    verde
## 2 Dinamarca  Claudio  1.59  3258 castanho
## 3    Brasil     Joao  1.63  6358     azul
## 4   Espanha     Jose  1.59  1258     azul
## 5   Holanda Gilberto  1.58  4555 castanho
\end{verbatim}

\textbf{Renomeando apenas as linhas}

\begin{Shaded}
\begin{Highlighting}[]
\KeywordTok{rownames}\NormalTok{(pesq) =}\StringTok{ }\KeywordTok{c}\NormalTok{(}\StringTok{"Obs 1"}\NormalTok{,}\StringTok{"Obs 2"}\NormalTok{,}\StringTok{"Obs 3"}\NormalTok{,}\StringTok{"Obs 4"}\NormalTok{,}\StringTok{"Obs 5"}\NormalTok{) }
\NormalTok{pesq}
\end{Highlighting}
\end{Shaded}

\begin{verbatim}
##           Var 1    Var 2 Var 3 Var 4    Var 5
## Obs 1       EUA  Alberto  1.93  5069    verde
## Obs 2 Dinamarca  Claudio  1.59  3258 castanho
## Obs 3    Brasil     Joao  1.63  6358     azul
## Obs 4   Espanha     Jose  1.59  1258     azul
## Obs 5   Holanda Gilberto  1.58  4555 castanho
\end{verbatim}

\textbf{Formas de ler bases de dados:} read.csv() - arquivos csv
read.xls() - arquivos excel read.spss() - arquivos spss read.mtp() -
arquivos minitab read.table() - arquivos txt read.delim() - leitura de
arquivos delimitados

\textbf{Dataframe de um arquivo csv}

\begin{Shaded}
\begin{Highlighting}[]
\NormalTok{df2 =}\StringTok{ }\KeywordTok{data.frame}\NormalTok{(}\KeywordTok{read.csv}\NormalTok{(}\DataTypeTok{file =} \StringTok{'C:/Users/junio/dframe.csv'}\NormalTok{,}
                          \DataTypeTok{header =} \OtherTok{TRUE}\NormalTok{, }\DataTypeTok{sep =} \StringTok{','}\NormalTok{))}
\NormalTok{df2}
\end{Highlighting}
\end{Shaded}

\begin{verbatim}
##   ID       Nome Idade    Admdate Diabete Status
## 1  1 Paciente 1    43 15/10/2015  Tipo 1   Ruim
## 2  2 Paciente 2    23 16/10/2015  Tipo 2    Bom
## 3  3 Paciente 3    56 23/10/2015  Tipo 2    Bom
## 4  4 Paciente 4    34 23/10/2015  Tipo 1   Ruim
## 5  5 Paciente 5    38 31/10/2015  Tipo 1  Medio
## 6  6 Paciente 6    37 28/10/2015  Tipo 1   Bom 
## 7  7 Paciente 7    41 27/10/2015  Tipo 1   Ruim
\end{verbatim}

\textbf{Primeiras linhas do dataframe}

\begin{Shaded}
\begin{Highlighting}[]
\KeywordTok{head}\NormalTok{(df2) }
\end{Highlighting}
\end{Shaded}

\begin{verbatim}
##   ID       Nome Idade    Admdate Diabete Status
## 1  1 Paciente 1    43 15/10/2015  Tipo 1   Ruim
## 2  2 Paciente 2    23 16/10/2015  Tipo 2    Bom
## 3  3 Paciente 3    56 23/10/2015  Tipo 2    Bom
## 4  4 Paciente 4    34 23/10/2015  Tipo 1   Ruim
## 5  5 Paciente 5    38 31/10/2015  Tipo 1  Medio
## 6  6 Paciente 6    37 28/10/2015  Tipo 1   Bom
\end{verbatim}

\textbf{Sumario do dataframe}

\begin{Shaded}
\begin{Highlighting}[]
\KeywordTok{summary}\NormalTok{(df2)}
\end{Highlighting}
\end{Shaded}

\begin{verbatim}
##        ID          Nome               Idade         Admdate         
##  Min.   :1.0   Length:7           Min.   :23.00   Length:7          
##  1st Qu.:2.5   Class :character   1st Qu.:35.50   Class :character  
##  Median :4.0   Mode  :character   Median :38.00   Mode  :character  
##  Mean   :4.0                      Mean   :38.86                     
##  3rd Qu.:5.5                      3rd Qu.:42.00                     
##  Max.   :7.0                      Max.   :56.00                     
##    Diabete             Status         
##  Length:7           Length:7          
##  Class :character   Class :character  
##  Mode  :character   Mode  :character  
##                                       
##                                       
## 
\end{verbatim}

\textbf{Algumas colunas da base de dados}

\begin{Shaded}
\begin{Highlighting}[]
\NormalTok{df2}\OperatorTok{$}\NormalTok{Diabete }\CommentTok{#coluna diabetes}
\end{Highlighting}
\end{Shaded}

\begin{verbatim}
## [1] "Tipo 1" "Tipo 2" "Tipo 2" "Tipo 1" "Tipo 1" "Tipo 1" "Tipo 1"
\end{verbatim}

\begin{Shaded}
\begin{Highlighting}[]
\NormalTok{df2}\OperatorTok{$}\NormalTok{Status }\CommentTok{#forma correta}
\end{Highlighting}
\end{Shaded}

\begin{verbatim}
## [1] "Ruim"  "Bom"   "Bom"   "Ruim"  "Medio" "Bom "  "Ruim"
\end{verbatim}

\textbf{Gerando um graficos de dispersao}

\begin{Shaded}
\begin{Highlighting}[]
\KeywordTok{summary}\NormalTok{(mtcars}\OperatorTok{$}\NormalTok{mpg)}
\end{Highlighting}
\end{Shaded}

\begin{verbatim}
##    Min. 1st Qu.  Median    Mean 3rd Qu.    Max. 
##   10.40   15.43   19.20   20.09   22.80   33.90
\end{verbatim}

\begin{Shaded}
\begin{Highlighting}[]
\KeywordTok{plot}\NormalTok{(mtcars}\OperatorTok{$}\NormalTok{mpg, mtcars}\OperatorTok{$}\NormalTok{disp) }\CommentTok{#grafico de dispersao entre duas variaveis}
\end{Highlighting}
\end{Shaded}

\includegraphics{knit_files/figure-latex/ars19-1.pdf}

\begin{Shaded}
\begin{Highlighting}[]
\KeywordTok{plot}\NormalTok{(mtcars}\OperatorTok{$}\NormalTok{mpg, mtcars}\OperatorTok{$}\NormalTok{wt) }\CommentTok{#outro grafico de dispersao entre duas variaveis }
\end{Highlighting}
\end{Shaded}

\includegraphics{knit_files/figure-latex/ars19-2.pdf}

\textbf{Combinando dataframes}

\begin{Shaded}
\begin{Highlighting}[]
\NormalTok{df3 =}\StringTok{ }\KeywordTok{merge}\NormalTok{(pesq,df2)}
\NormalTok{df3}
\end{Highlighting}
\end{Shaded}

\begin{verbatim}
##        Var 1    Var 2 Var 3 Var 4    Var 5 ID       Nome Idade    Admdate
## 1        EUA  Alberto  1.93  5069    verde  1 Paciente 1    43 15/10/2015
## 2  Dinamarca  Claudio  1.59  3258 castanho  1 Paciente 1    43 15/10/2015
## 3     Brasil     Joao  1.63  6358     azul  1 Paciente 1    43 15/10/2015
## 4    Espanha     Jose  1.59  1258     azul  1 Paciente 1    43 15/10/2015
## 5    Holanda Gilberto  1.58  4555 castanho  1 Paciente 1    43 15/10/2015
## 6        EUA  Alberto  1.93  5069    verde  2 Paciente 2    23 16/10/2015
## 7  Dinamarca  Claudio  1.59  3258 castanho  2 Paciente 2    23 16/10/2015
## 8     Brasil     Joao  1.63  6358     azul  2 Paciente 2    23 16/10/2015
## 9    Espanha     Jose  1.59  1258     azul  2 Paciente 2    23 16/10/2015
## 10   Holanda Gilberto  1.58  4555 castanho  2 Paciente 2    23 16/10/2015
## 11       EUA  Alberto  1.93  5069    verde  3 Paciente 3    56 23/10/2015
## 12 Dinamarca  Claudio  1.59  3258 castanho  3 Paciente 3    56 23/10/2015
## 13    Brasil     Joao  1.63  6358     azul  3 Paciente 3    56 23/10/2015
## 14   Espanha     Jose  1.59  1258     azul  3 Paciente 3    56 23/10/2015
## 15   Holanda Gilberto  1.58  4555 castanho  3 Paciente 3    56 23/10/2015
## 16       EUA  Alberto  1.93  5069    verde  4 Paciente 4    34 23/10/2015
## 17 Dinamarca  Claudio  1.59  3258 castanho  4 Paciente 4    34 23/10/2015
## 18    Brasil     Joao  1.63  6358     azul  4 Paciente 4    34 23/10/2015
## 19   Espanha     Jose  1.59  1258     azul  4 Paciente 4    34 23/10/2015
## 20   Holanda Gilberto  1.58  4555 castanho  4 Paciente 4    34 23/10/2015
## 21       EUA  Alberto  1.93  5069    verde  5 Paciente 5    38 31/10/2015
## 22 Dinamarca  Claudio  1.59  3258 castanho  5 Paciente 5    38 31/10/2015
## 23    Brasil     Joao  1.63  6358     azul  5 Paciente 5    38 31/10/2015
## 24   Espanha     Jose  1.59  1258     azul  5 Paciente 5    38 31/10/2015
## 25   Holanda Gilberto  1.58  4555 castanho  5 Paciente 5    38 31/10/2015
## 26       EUA  Alberto  1.93  5069    verde  6 Paciente 6    37 28/10/2015
## 27 Dinamarca  Claudio  1.59  3258 castanho  6 Paciente 6    37 28/10/2015
## 28    Brasil     Joao  1.63  6358     azul  6 Paciente 6    37 28/10/2015
## 29   Espanha     Jose  1.59  1258     azul  6 Paciente 6    37 28/10/2015
## 30   Holanda Gilberto  1.58  4555 castanho  6 Paciente 6    37 28/10/2015
## 31       EUA  Alberto  1.93  5069    verde  7 Paciente 7    41 27/10/2015
## 32 Dinamarca  Claudio  1.59  3258 castanho  7 Paciente 7    41 27/10/2015
## 33    Brasil     Joao  1.63  6358     azul  7 Paciente 7    41 27/10/2015
## 34   Espanha     Jose  1.59  1258     azul  7 Paciente 7    41 27/10/2015
## 35   Holanda Gilberto  1.58  4555 castanho  7 Paciente 7    41 27/10/2015
##    Diabete Status
## 1   Tipo 1   Ruim
## 2   Tipo 1   Ruim
## 3   Tipo 1   Ruim
## 4   Tipo 1   Ruim
## 5   Tipo 1   Ruim
## 6   Tipo 2    Bom
## 7   Tipo 2    Bom
## 8   Tipo 2    Bom
## 9   Tipo 2    Bom
## 10  Tipo 2    Bom
## 11  Tipo 2    Bom
## 12  Tipo 2    Bom
## 13  Tipo 2    Bom
## 14  Tipo 2    Bom
## 15  Tipo 2    Bom
## 16  Tipo 1   Ruim
## 17  Tipo 1   Ruim
## 18  Tipo 1   Ruim
## 19  Tipo 1   Ruim
## 20  Tipo 1   Ruim
## 21  Tipo 1  Medio
## 22  Tipo 1  Medio
## 23  Tipo 1  Medio
## 24  Tipo 1  Medio
## 25  Tipo 1  Medio
## 26  Tipo 1   Bom 
## 27  Tipo 1   Bom 
## 28  Tipo 1   Bom 
## 29  Tipo 1   Bom 
## 30  Tipo 1   Bom 
## 31  Tipo 1   Ruim
## 32  Tipo 1   Ruim
## 33  Tipo 1   Ruim
## 34  Tipo 1   Ruim
## 35  Tipo 1   Ruim
\end{verbatim}

\end{document}
